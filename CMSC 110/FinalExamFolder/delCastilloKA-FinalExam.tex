\documentclass{article}

\renewcommand{\labelenumii}{\roman{enumii}}
\usepackage{geometry}
\usepackage{natbib}
\geometry{
  a4paper,
  total={170mm, 257mm},
  left=20mm,
  top=20mm
}

\begin{document}
\begin{center}
  \textbf{CMSC 110 Final Examination}
  \\~\\
  \raggedright Name: Kyle Adrian del Castillo \hfill
  \centering Student\#:2020-04403 \hfill
  \raggedleft Date: 07/01/2023: 18:00
\end{center}

\subsection*{Questions}

\hfill
\begin{enumerate}
  \item \textbf{Please explain briefly the differences between a conference paper and a journal paper.}
  \subitem A conference paper is a brief research work of 4-5 pages peer reviewed for 1-3 months and accepted by a conference review panel, and is more suited towards beginners. The paper is registered and presented during the conference by the authors, then published as a part of the conference proceedings.
  \subitem A journal paper on the other hand is a full-length paper of 8-10 pages of high complexity. It undergoes rigorous review of 1-12 months and may be revised several times, suitable for expert researchers. If accepted by the journal, it has no registration fee but the authors will have to pay to make their paper open access, it is then posted on the journal’s website.

  \item \textbf{Poster papers and conference papers are both presented in a research conference, which can be local or international. Describe the way these two types of paper are presented in a research conference.}
  \subitem Poster papers are presented during a poster session at a conference, where there are several posters summarizing a research presented by its author where people can ask questions and glance at posters. Conference papers are presented orally for a specified time period which depends on the conference stipulations.
  \subitem In a study of the publication patterns of the Congress of Neurological Surgeons and American Association of Neurological Surgeons, about 41\% of oral presentations and 29\% of poster presentations were eventually published, with 98\% being published within 5 years after presentation at the conference \citep{patel_publication_2011}.

  \item \textbf{Briefly explain the important items to remember when preparing a very good poster paper for presentation in a research conference.}
    The general guidelines of making a good poster paper include: 
    \begin{enumerate}
      \item Cutting down on text and making sure to include only important points of information.
      \item Structuring your poster to contain a flow of events as if to "tell a story".
      \item Minimizing clutter on your poster to "let it breathe".
      \item Choosing the right color palette.
      \item Choosing the right font and font size.
      \item Use a concise, interesting title.
      \item Simplify figures and make them more descriptive.
    \end{enumerate}    
    \subitem However, \cite{morrison_how_2020} argues in his informative video that the traditional poster design has not been updated for more than 20 years and presents elements of User Experience (UX) Design to smoothen the cognitive load posters for passing readers. 
    \subitem The key takeaway is to present the main finding along with an optional graphic or figure in the middle of your poster in huge font size as if to present in a billboard, and add more details about the research in smaller font sizes to the sides in bullet points, easing the transfer of information to the reader \citep{morrison_how_2020}.
 
  \item \textbf{Journal papers and technical reports convey new things to some interested readers. Please describe the similarities and differences between these two types of documents.}
  \subitem A technical report is a short article that gives a brief description of a method, technique, or procedure, with a certain limit of words and figures depending on the stipulations of a journal and is usually less than that of a journal article \citep{ng_writing_2010}. A journal article on the other hand, is a full-length research work. 
  \subitem The two are similar in that they present information in a systematic and informative manner, however technical reports are usually not peer-reviewed and only focus on presenting novel methodologies or results from novel techniques, methods, or equipment \citep{ng_writing_2010}. Journal articles however, are extensively reviewed and are only limited to topics accepted by the the publishing journal.
  \newpage
  \item \textbf{Briefly explain the important items to remember when preparing a very good technical report.}
  \item[] The guidelines of making a professional technical report that takes into account the its reader and purpose include:
  \begin{enumerate}
    \item Making complex concepts clear, clarity is key.
    \item Design your report for a busy user and place key information where the reader would expect it to be.
    \item Help your reader understand your findings by communicating what you did, what you found, what your findings mean, and what are you conclusions or recommendations.
    \item Tailor your report to your audience by accounting for the amount of content you include, organizing the report body logically, and accounting for the jargon that the audience understands.
  \end{enumerate}

  \item \textbf{Journal papers and patent documents convey new things to some interested readers. Please describe the similarities and differences between these two types of documents.}
  \subitem The patent document contains bibliographic, technical, and legal information describing the patent, and is published to the official government body that handles patent applications and stipulates patent law.
  \subitem The journal paper and patent document are similar in that they are professional documents that contain technical information, however they are vastly different in that patent documents contain policy-defining information like the claims of the patent and drawings describing the patent among others. Patents are limited by its country's patent law regarding what are patentable inventions (and what are not).
  \subitem Journal articles however can contain any kind of information and is only limited by what the publishing journal stipulates.

  \item \textbf{In your own words, briefly but sufficiently discuss the difference between a “Patent” and “Utility Model.”}
  \subitem Patents require an inventive step while a utility model does not. This is because a Utility model is designed to protect inventions that are not "sufficiently inventive" to meet the standards for patent application. A utility model is also entitled to seven years of protection from the date of filing without possibility of renewal.
  \subitem Patents, however, are granted 20 years of protection and requires extensive examination after publication. A utility model is immediately registered upon filing after meeting all formality requirements.

  \item \textbf{Discuss the three criteria that must be met in order for a patent application be granted.}
  \item[] To three standards that must be met to be considered for patentability are:
  \begin{enumerate}
    \item Novelty
    \subitem This means that the invention must not have been made public before the date of the application.
    \item Inventive Step
    \subitem This means that the product or process must be an inventive solution. The product or process must meet a certain threshold of inventiveness that is not an obvious solution.
    \item Industrial Applicability
    \subitem It must be possible to manufacture or materialize the novel invention. For example, a new kind of playing card that is easier to hold than existing cards is patentable, but an idea for a new card game is not.
  \end{enumerate}
  \subitem For a patent application to be accepted, they undergo extensive examination based on a certain threshold of these three criteria. 

  \item \textbf{In patent search, discussed the advantages and the disadvantages of using the International Patent Classification (IPC).}
  \subitem The limitations of IPC is that patent documents are readily available in French and English and may cause a language barrier for other inventors. Alternatives like the Cooperative Patent Classification (CPC) and the Japanese File Index (FI) system are already integrated in the IPC Official Publication, and most countries globally make use of the IPC.
  
  \item \textbf{Explain why the “Claims” is the most important part of any patent document.}
  \subitem The "Claims" section of a patent document that defines the scope of legal protection that the invention enjoys. Each claim explicitly defines the boundaries of the patented invention which means that if the invention contains a feature which is not listed under its claims, that feature is not protected by law. 
  \subitem This is the most important part of any patent document because this is where the parts of the invention that the law protects are defined.

\end{enumerate}

\bibliographystyle{agsm}
\bibliography{CMSC110exam}

\end{document}
